\documentclass[11pt,a4paper]{article}

\usepackage[left=2cm,text={17cm,24cm},top=3cm]{geometry}
\usepackage[english]{babel}
\usepackage[utf8]{inputenc}
\usepackage[T1]{fontenc}
\usepackage{indentfirst}
\usepackage{hyperref}

% URL
\usepackage{url}
\def\UrlBreaks{\do\/\do-}
\usepackage{breakurl}
\usepackage[table,xcdraw]{xcolor}
\usepackage{graphicx}
\usepackage{booktabs}
\usepackage[bottom]{footmisc} % footnotes are now displayed at the bottom of the page

\usepackage{algpseudocode}
\usepackage{algorithm}




% IMG
\usepackage{graphicx}
\graphicspath{{.}}


% #################################################################################################
\begin{document}
% #################################################################################################


% #################################################################################################
% TITLEPAGE
\begin{titlepage}
    \begin{center}
        \Huge
        \textsc{
            Faculty of Information Technology\\
            Brno University of Technology
        }
        \vspace{80px}
        \begin{figure}[!h]
            \centering
            \includegraphics[scale=0.3]{img/vutbr-fit-logo.eps}
        \end{figure}
        \\[15mm]
        \Huge{
            \textbf{
                ISA
            }
        }
        \\[1.5mm]
        \huge{
            \textbf{
                Network Applications and Network Administration
            }
        }
        \\[2.5em]
        \LARGE{
            \textbf{
                Export DNS Information\\
                By Protocol Syslog
            }
        }
        \vfill
    \end{center}
        \Large{
            \hfill\\
            Attila Lakatos (xlakat01) \hfill \today
        }

\end{titlepage}


% #################################################################################################
% CONTENT


\setlength{\parskip}{0pt}
\hypersetup{hidelinks}\tableofcontents
\setlength{\parskip}{0pt}

\newpage %#########################################################################################

\section{Abstract}

The primary goal of this project is to analyze and print out the description of both incoming and outgoing DNS packet data from a valid pcap file. It also allows the user to display DNS packets being transmitted or recieved over a network which the computer is attached. It is also possible to use dnx-export for the specific purpose of summarizing and sending the communications to syslog server. First of all, this document gives a brief introduction about DNS(Domain Name System). After that, we will cover what packet sniffing is as well as how to analyze the data packets that have been captured and analyzed on a network. Then it follows a detailed description about the implementation part describing data parsing on different layers. This will solidify a basic understanding of packet sniffing. The most troublesome parts and difficult steps which made the programming part more complicated will be disscussed in this documentation. Finally, we will encompass the conclusions and the discussion of the results from the project. The authors point of view towards the task is included as well.

\section{Introduction}

The Domain Name System (DNS) is an application–layer protocol. This protocol implements the DNS naming service, which is the naming service used on the Internet. This documentation assumes that the reader has some familiarity with network programming. DNS provides a so-called, Name-To-Address Resolution. It is a worldwide hierarchy of computers on the Internet. Although, the basic funcionality of DNS is simplier. Name-to-adress, also known as mapping, is a process of finding the IP adress of a computer in a database by using its hostname.

To use DNS, we send query to a DNS server. This query contains the domain name which we are looking for. DNS server tries to find that domain name's IP adress in its data store. In some cases it finds it, after that it returnes it. If the IP has not been found yet, the server will forward the query to another DNS name server. The process is repeated until the IP is found\cite{DNS}. DNS is deeply documented in RFC 1035 \cite{RFC1035}.

Each subsection in this section analyzes a table provided by networksorcery organization \cite{INFORMATIONS}.

\subsection{DNS header in nutshell}

The following figure describes what kind of informations does DNS header contain. The very first row highlighted by gray color indicates the position of each bit. It is followed by 4 constantly given 32 bits long fields. Each field will be described accurately in the next subsections. Rows having green background color represent fields with variable sizes. \\

\resizebox{\textwidth}{!}{%
\renewcommand{\arraystretch}{2}
\begin{tabular}{cccccccccccccccccccccccccccccccc}
\hline
\rowcolor[HTML]{EFEFEF}
\textbf{00} & \textbf{01} & \textbf{02} & \textbf{03} & \textbf{04} & \textbf{05} & \textbf{06} & \textbf{07} & \textbf{08} & \textbf{09} & \textbf{10} & \textbf{11} & \textbf{12} & \textbf{13} & \textbf{14} & \textbf{15} & \textbf{16}                                     & \textbf{17} & \textbf{18} & \textbf{19} & \textbf{20} & \textbf{21}                                     & \textbf{22}                                     & \textbf{23}                                     & \textbf{24}                                     & \textbf{25}                                    & \textbf{26}                                     & \textbf{27}                                     & \textbf{28} & \textbf{29} & \textbf{30} & \textbf{31} \\ \hline
\rowcolor[HTML]{CD9934}
\multicolumn{16}{|c|}{\cellcolor[HTML]{CD9934}IDENTIFICATION}                                                                                                                                                                 & \multicolumn{1}{c|}{\cellcolor[HTML]{CD9934}QR} & \multicolumn{4}{c|}{\cellcolor[HTML]{CD9934}OPCODE}   & \multicolumn{1}{c|}{\cellcolor[HTML]{CD9934}AA} & \multicolumn{1}{c|}{\cellcolor[HTML]{CD9934}TC} & \multicolumn{1}{c|}{\cellcolor[HTML]{CD9934}RD} & \multicolumn{1}{c|}{\cellcolor[HTML]{CD9934}RA} & \multicolumn{1}{c|}{\cellcolor[HTML]{CD9934}Z} & \multicolumn{1}{c|}{\cellcolor[HTML]{CD9934}AD} & \multicolumn{1}{c|}{\cellcolor[HTML]{CD9934}CD} & \multicolumn{4}{c|}{\cellcolor[HTML]{CD9934}RCODE}    \\ \hline
\rowcolor[HTML]{CD9934}
\multicolumn{16}{|c|}{\cellcolor[HTML]{CD9934}TOTAL QUESTIONS}                                                                                                                                                                & \multicolumn{16}{c|}{\cellcolor[HTML]{CD9934}TOTAL ANSWERS RRs}                                                                                                                                                                                                                                                                                                                                                                                                                                                              \\ \hline
\rowcolor[HTML]{CD9934}
\multicolumn{16}{|c|}{\cellcolor[HTML]{CD9934}TOTAL AUTHORITY RRs}                                                                                                                                                            & \multicolumn{16}{c|}{\cellcolor[HTML]{CD9934}TOTAL ADDITIONAL RRs}                                                                                                                                                                                                                                                                                                                                                                                                                                                           \\ \hline
\rowcolor[HTML]{67FD9A}
\multicolumn{32}{|c|}{\cellcolor[HTML]{67FD9A}QUESTIONS {[}{]}}                                                                                                                                                                                                                                                                                                                                                                                                                                                                                                                                                                                                                                                                                              \\ \hline
\rowcolor[HTML]{67FD9A}
\multicolumn{32}{|c|}{\cellcolor[HTML]{67FD9A}ANSWERS RRs {[}{]}}                                                                                                                                                                                                                                                                                                                                                                                                                                                                                                                                                                                                                                                                                            \\ \hline
\rowcolor[HTML]{67FD9A}
\multicolumn{32}{|c|}{\cellcolor[HTML]{67FD9A}AUTHORITY RRs {[}{]}}                                                                                                                                                                                                                                                                                                                                                                                                                                                                                                                                                                                                                                                                                          \\ \hline
\rowcolor[HTML]{67FD9A}
\multicolumn{32}{|c|}{\cellcolor[HTML]{67FD9A}ADDITIONAL RRs {[}{]}}                                                                                                                                                                                                                                                                                                                                                                                                                                                                                                                                                                                                                                                                                         \\ \hline
\end{tabular}%
}

\newpage

\subsection{DNS header section format}
\label{HEADER}

\begin{enumerate}
   \item Identification - This identifier is copied the corresponding reply and can be used by the requester to match up replies to outstanding queries.
   \item Flags
   \begin{itemize}
     \item QR - Indicates if it is a query or response. In this project we will deal only with responses.
     \item Opcode - Query type.
     \item AA - Specifies that the responding name server is an authority.
     \item TC - 1 indicates that only the first 512 bytes of the reply was returned.
     \item RD - Recursive query support is optional.
     \item RA - Recursive query support is/is not available in the name server
     \item Z - Reserved for future use.
     \item AD - It should be set only if all data in the response has been cryptographically verified or otherwise meets the server's local security policy.
     \item CD - Checking disabled.
     \item Rcode - Response code.
   \end{itemize}
   \item Total questions - Number of questions
   \item Total answer RRs - Number of answers
   \item Total authority RRs - Number of authority RRs
   \item Total additional RRs - number of additional RRs
\end{enumerate}


\subsection{DNS query section format}
\label{QUERY}

\input{dnsquery.tex}
\begin{enumerate}
    \item Query name - Domain name represented as a sequence of labels. It will be deeply explained in the implementation part.
    \item Type - Specifies the type of the query.
    \item Class - Specifies the class of the query.
\end{enumerate}
\newpage

\subsection{DNS resource record format}
\label{RESPONSE}

\resizebox{\textwidth}{!}{%
\renewcommand{\arraystretch}{2}
\begin{tabular}{cccccccccccccccccccccccccccccccc}
\hline
\rowcolor[HTML]{EFEFEF}
\textbf{00} & \textbf{01} & \textbf{02} & \textbf{03} & \textbf{04} & \textbf{05} & \textbf{06} & \textbf{07} & \textbf{08} & \textbf{09} & \textbf{10} & \textbf{11} & \textbf{12} & \textbf{13} & \textbf{14} & \textbf{15} & \textbf{16}                                     & \textbf{17} & \textbf{18} & \textbf{19} & \textbf{20} & \textbf{21}                                     & \textbf{22}                                     & \textbf{23}                                     & \textbf{24}                                     & \textbf{25}                                    & \textbf{26}                                     & \textbf{27}                                     & \textbf{28} & \textbf{29} & \textbf{30} & \textbf{31} \\ \hline

\rowcolor[HTML]{67FD9A}
\multicolumn{32}{|c|}{\cellcolor[HTML]{67FD9A}NAME}                                                                                                                                                                                                                                                                                                                                                                                                                                                                                                                                                                                                                                                                                         \\ \hline

\rowcolor[HTML]{CD9934}
\multicolumn{16}{|c|}{\cellcolor[HTML]{CD9934}TYPE}  & \multicolumn{16}{c|}{\cellcolor[HTML]{CD9934}CLASS }                                                                                                                                                                                                                                                                                                                                                                                                        \\ \hline

\rowcolor[HTML]{67FD9A}
\multicolumn{32}{|c|}{\cellcolor[HTML]{CD9934}TTL}


\\ \hline

\rowcolor[HTML]{CD9934}
\multicolumn{16}{|c|}{\cellcolor[HTML]{CD9934}RDATA LENGTH}  & \multicolumn{16}{c|}{\cellcolor[HTML]{67FD9A}RDATA }                                                                                                                                                                                                                                                                                                                                                                                                        \\ \hline


\end{tabular}%
}

\begin{enumerate}
    \item Name - Domain name to which this resource record pertains.
    \item Type - This field specifies the meaning of the data in the RDATA field.
    \item Class - Specifies the class of the data in the RDATA field.
    \item TTL - The time interval (in seconds) that the resource record may be cached before it should be discarded.
    \item RD length - Specifies the length in octets of the RDATA field.
    \item RData - The format of this information varies according to the TYPE and CLASS of the resource record. For more information plese visit RFC 1035\cite{RFC1035}.
\end{enumerate}

\subsection{Information compression}
To understand how QNAME is represented(same goes for NAME and RDATA) we need to know how a domain name is stored in a DNS packet. The domain name is represented in the form of labels separeted by dots. It can be represented as either a sequence of labels ending in a zero octet, a pointer or a sequence of labels ending with a pointer. The purpose of using the foregoing representations are needed in order to reduce the size of messages, the domain name system utilizes a compressions scheme which eliminates duplicated domain names in a message. In some cases an entire domain name or a list of labels can be replaced with a specific pointer.
As we dive deeper into dns name interpreting, we can see that the first two bits tells us if it is a label or pointer:
\begin{itemize}
    \item 0b00 - Labels must begin with two zero bits, because labels are restricted to 63 octets or less.
    \item 0b01 - Reserved for future use.
    \item 0b10 - Reserved for future use.
    \item 0b11 - Hexidecimal representation is '0x0c'. This allows a pointer to be distinguished from a label, since a label must begin with 2 zero bits. Remaining bits are being used to calculate where the pointer points to.
\end{itemize}

\newpage
\section{Implementation}
C++ was choosen as the implementation language because it has object-oriented and generic programming features, while also providing facilities for low-lever memory programming. Furthermore, allows to work with strings in a more efficient way. In the next subsections, we will discuss how the program was implemented by progressing gradually from one stage to the next.

As I mentioned earlier, some DNS resource records can vary according to the type and class of its record. These constants can be found in constants.h header file. It also contains error values returned by our program while encountering an invalid state during compilation.

First of all, the whole process starts with parsing input arguments. After that, each parameter is stored in a structure for further application. Functions connected to argument checking take place in argparer.cpp. In case of an error occured during compilation a brief message is displayed by a program that utilizes a command-line interface for execution. For further information man(dns-export) is available too.

At this point, program will continue in one of the following ways. A valid file has been provided by a user to parse packets from it or an interface flag was specified which causes it to read packets from a network interface rather than read from a saved packet. Reading packets from a network interface may require that you have special privileges; see the pcap man page for details. Reading a saved packet file doesn't require special privileges. These two methods share same algorithms, they only differ at the begining where I define either an online\footnote{online: read packets from network interface} or offline\footnote{offline: read from a saved packet file} method. Fortunately, pcap header file provides us some useful functions to parse data from dns packets. I will be sure to describe concepts in greater detail.

We begin by identifying which interface we want to sniff on. We can either define an interface as a string or we can sniff on all interfaces by means of 'ANY'. Next, pcap needs to be initialized. We differentiate between devices using file handles, just like an ordinary ones for file handling. The above listed steps only refered to live sniffing. Then, we must create a rule set not to parse unneccssary packets. This can be achieved by setting port to 53. Compiling and aplying is also required. The rule set is kept in a string, and is converted into a format that pcap can read (hence compiling it). Finally, we tell pcap to enter it's primary execution loop. In this state, pcap waits until it has received however many packets we want it to. Every time it gets a new packet in, it calls another function that we have already defined.

From now on, things will constantly get more and more complicated. First of all, pcap\_datalink(p) returns the link-layer header type for the live capture or "savefile" specified by p. We only deal with ethernet and linux cooked capture. It gives us information about which internet protocol version(IPv4 or IPv6) does it support. After successfully parsing link-layer header, we can move on to internet protocol version. Size of IPv6 is constantly defined, while IPv4 has a special field containg additional information connected to its size. Penultimate task is to successfully identify and handle the TCP/UDP layer. Most likely DNS packets will arive as UDP ones but in some cases, such as longer and more detailed answers can appear. They can be handled via TCP. UDP header size is set in advance, contrary to TCP where it need to be read from a special field called th\_off (the length of the TCP header is always a multiple of 32 bits).

At the current state we have a pointer which tells us where does DNS query/response start. By the help of transaction ID we can identify which query is connected to which response. This property is properly used while tracking down TCP packets. DNS packets using TCP protocol contain an additional 2 byte length field. There are some obstacles between us and DNS answers rigth now, to get right there we have to skip Transaction ID, Flags, Questions, Answers, Authority RRs and Additional RRs; as I mentioned earlier their size is constantly defined. Only fields that remain are queries, for this purpose I have defined a so-called \textit{jumpToDnsAnswers()} function, which skips the whole unnecessary content. More detailed description by using pseudocode is also availabe in section \ref{ALGORITHMS}.

As the project task requires, the very first indispensable information containing a name needs to be analysed and stored for further processing. It is achieved when dnsNameToString() returnes a valid name, it also ensures pointer will point to the next field. The type field indicates how many additional fields does DNS response contain. This is solved by calling \textit{parseDNSdata()} function which returnes a string containing additional resource record informations. Every single information related to resource records is being stored in a C++ list of strings for further processing. More detailed informations about above listed functions can be found in section \ref{ALGORITHMS}.

\subsection{Signal handling}
Signals are the interrupts delivered to a process by the operating system which can terminate a program prematurely or can affect their interior behavior in an elegent way. If dns-export encounters a SIGUSR1 signal, typically generated by the user calls \textit{printOutStatistics()} function which prints out data according to every single gathered information till now.

When a user specifies a syslog server, gathered information is also transferred to syslog server via UDP protocol. It consists of two steps, creating syslog header format plus creating syslog message format. The above stated functions are defined in syslog.cpp. The \textit{alarm()} function causes the system to generate a SIGALRM signal for the process after the number of realtime seconds specified by seconds have elapsed. SIGALRM signals are handled by \textit{sendToSyslog()} functions, which casues an immediate data transfer to the specified syslog server.

\subsection{Limitations}
\label{LIMITATIONS}
It is hard to define what parts of the assignment have been left out. Reassembled TCP packet analysing, parsing and information gathering are not part of final program, unlike simplier TCP packets.

\subsection{Testing}
\label{TESTING}
Thanks to available pcap files provided by Ing. Petr Matoušek, Ph.D., M.A. via information system I was able to test the main parts of DNS pacet data parsing. There are 2 tools which need to be mentioned. Wireshark is a network protocol-analyzer, it helped me to see what was happening on a certain network or in a valid pcap file. It has a bunch of features, including packet observation, filtering and analysing. DIG(domain information gropper), an administration command-line tool was also used for querying Domain Name System servers.


\section{Algorithms}
\label{ALGORITHMS}

\label{skipDNSQuery}
\begin{algorithm}
    \caption{Skip DNS query section}
    \begin{algorithmic}

    \Procedure{jumpToDnsAnswers}{$label$}
        \If {label is NULL}
            \State return NULL
        \EndIf
        \If {label is a pointer}
            \State return label+2
        \EndIf
        \While {label is not char with 0 ASCII CODE}
            \State Increase label pointer
        \EndWhile
        \State return label+1
    \EndProcedure
    \end{algorithmic}
\end{algorithm}

\label{DNSNameToString}
\begin{algorithm}
    \caption{DNS name to string representation}
    \begin{algorithmic}

    \Procedure{dnsNameToString}{$label, payload, end$}
        \State $ string \gets "" $
        \While {adress of label < adress of end AND label is not a char with 0 ASCII CODE}
            \If {label is a pointer}
                \State $ tmp \gets payload+characterWhereLabelPoints $
                \While {adress of tmp < adress of end AND tmp is not a char with 0 ASCII CODE}
                    \If {tmp is a pointer}
                        \State $ tmp \gets payload+characterWhereLabelPoints $
                    \EndIf
                    \State $ len \gets label $
                    \State $ tmp \gets tmp $ + 1
                    \State $ string \gets string $ + len number of characters
                    \State $ string \gets string$ + '.'
                \EndWhile
                \State $ label \gets label $ + 1
            \Else
                \State $ len \gets label $
                \State $ label \gets label $ + 1
                \State $ string \gets $ len number of characters
                \State $ string \gets string$ + '.'
            \EndIf

        \EndWhile
        \State return string
    \EndProcedure
    \end{algorithmic}
\end{algorithm}

\newpage


\section{Conclusion}

A couple of scientific reports have been surveyed, mainly connected with packet parsing provided by different sources on the internet. The most troublesome part was to precisely interpret the content of DNS labels and pointers due to lack of available materials connected to this topic. I have to admit this project belongs to the most interesting ones which I had to deal with in my studies at VUT FIT. As I mentioned earlier in subsection \ref{LIMITATIONS}, some objectives, such as parsing reassembled TCP packets have been left out due to lack of time.

\newpage %#########################################################################################

\makeatletter
\makeatother
\bibliographystyle{englishiso}
\begin{flushleft}
    \bibliography{quotation}
\end{flushleft}


% #################################################################################################
\end{document}
% #################################################################################################
